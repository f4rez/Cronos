\documentclass[12pt,a4paper]{article}
\usepackage[utf8]{inputenc}
\usepackage{amsmath}
\usepackage{amsfonts}
\usepackage{amssymb}
\author{Alfred Yrelin, Josef Svensson, Philip Åkerfeldt}
\title{Undefined Title Project}
\begin{document}
\maketitle

\section{Förord}
\textit{Tack till}
\tableofcontents
\section{Introduktion}
\textit{Sammanfattande text}
Målet med det här projektet är att utveckla ett mobilspel. Spelet går ut på att placera givna händelser i rätt årtalsordning och genom detta vinna över sin motståndare. Spelare har möjlighet att utmana sina vänner, men ska också ha möjlighet att spela mot slumpvalda motståndare. 
\subsection{Syfte}
\textit{Varför tror vi att det här är ett bra projekt?
Vilka kunskaper kan vi få genom detta?
Vilka kunskaper som vi redan lärt oss kan vi få användning av?
Hur bidrar det här till vår examen?}


Det här är ett bra projekt eftersom det binder samman kunskaper från flera tidigare kurser. Det är också en stor utmaning att faktiskt göra en produkt som ska bli så pass klar att en riktig användare kan ha nytta av den. Slutfinishen i de projekt som genomförts i tidigare kurser är någonting som oftast inte varit så viktigt. I det här projektet blir det därför en stor och nyttig utmaning att faktiskt färdigställa applikationen så pass att den blir användbar på riktigt. 
\subsection{Avgränsningar}
\textit{Vilka delar av applikationen är inte implementerade än, vilka valdes bort?}
\section{Förberedelser}
\subsection{Marknadsundersökning}
\textit{Tror andra att det här är en bra appidé?}
\subsection{Projektplan}
\textit{Hur tänker vi strukturera upp arbetet? Scrum?
[FANTASTISK BILD HÄR]}

En grov första planering av projektet ser ut så här:
\begin{enumerate}
\item Göra upp spelmekanik och lära oss Go samt androidutveckling.
\item Sätta upp grundläggande kommunikation mellan server och klient, samt en bas för vardera.
\item Utveckla betaversion
\item Betatesta appen
\item Utvärdera spelet med tekniker från människa-datainteraktionsområdet.
\item Förbättringsarbete enligt användarkritik och expertutvärdering.
\item Lansera applikationen
\end{enumerate}
\subsection{Systemdesign}
\textit{Här visar vi hur systemet är tänkt att byggas. Gärna ett flödesdiagram över de olika modulerna. Gamla och nyare versioner kan jämföras.}
\subsubsection{Gränssnitt}

\subsection{Tekniska svårigheter}

\section{Teori}
\textit{Finns det andra projekt som lyckats? Borde detta lyckas? Är konceptet väl beprövat?}
\section{Implementation}
Applikationen består av två stora delar, en serverdel och en klientdel. Serverdelen är skriven i programspråket GO\footnote{http://golang.org/} och körs i Googles \textit{App Engine}. Klienten är en Androidapplikation skriven i Java.
\subsection{Android}
\subsection{Server - GO!}
\subsection{Problem}
\textit{Intressanta problem som uppstått under vägens gång}
\section{Resultat}
\textit{Hur blev appen i jämförelse mot hur vi tänkte från början? Vilka delar har lagts till och vilka saknas? Varför har det ändrats från den första skissen? Är den klar?}
\section{Diskussion}
\textit{Finns det något intressant att reflektera över?}
\subsection{Grupparbetet}
\textit{Gick grupparbetet bra?}
\section{Källor}
\appendix
\end{document}