\documentclass[a4paper,11pt]{article}
\usepackage{fullpage}
\usepackage[utf8]{inputenc}
\usepackage[british]{babel}
\usepackage{amsmath}
\usepackage{amssymb}
\usepackage{amsthm}
\usepackage{color}
\usepackage{float}
\usepackage{listings}
\usepackage{fontenc}
\usepackage[hidelinks]{hyperref}
\usepackage[pdftex]{graphicx}
\usepackage{multirow}
\usepackage{gensymb}




\title{\textbf{Projektbeskrivning \\
    Uppsala University -- Spring 2015 \\
      }}

\author{
Josef Svensson\\
\and
Alfred Yrelin\\
\and
Philip \AA kerfeldt\\
}


\date{\today}

\begin{document}
\maketitle
\newpage
\section{Projektbeskrivning}
Som projekt i informationsteknologi har vi ett koncept för ett spel i åtanke. Konceptet vi tänkte utveckla till en produkt är ett turbaserat multiplayerspel som först och främst ska vara till Android eventuellt, om tiden tillåter, även till iOS.\\ Spelet har en ganska enkel uppbyggnad där man kan spela själv mot datorn eller så får utmana en vän på en match. Matcherna går ut på att sortera händelser från historien i rätt ordning. Man ges en händelse t.ex. 'Olof Palme mördas' som referenshändelse (läs starthändelse) och får sedan en ny händelse kanske 'genghis khan dör' och ska bestämma om det hände före eller efter Palme mördades. Får man rätt ges en ny händelse och den ska sorteras gentemot de två föregående händelserna. Exakt hur matcherna kommer ta sin form har vi inte bestämt, men konceptet följer mallen som beskrivit ovan. Det kan även utökas till att även sortera andra saker gentemot varandra t.ex. Höjd på byggnader, mest vunna VM guld osv.
Spelet kommer ha ett antal olika spellägen som användaren kan välja mellan. De lägen vi kommit fram till just nu är:
\begin{itemize}
\item[•] Först till en viss poäng
\item[•] Vinna flest rundor under en match
\item[•] Infinity mode
\item[-] Detta är ett spelläge som passar för användare som är riktiga duktiga i multiplayer eller om man vill testa sina kunskaper mot datorn. Spelläget liknar sudden death och man spelar till den som först får ett fel.
\item[•] Rapid mode
\item[-] Detta spelläge passar för den som vill spela en kort runda pga bristande tid. Matcherna är av bestämd längd och har ett bestämt antal omgångar per match. Storleken på matcherna kan avgöras vid valet av Rapid Mode. \\ 
\end{itemize}

\noindent
Till denna applikation behöver vi skapa en server + klient. Till serversidan har vi tänkt använda Google App Engine och använda språket GO, detta för att serven ska vara otroligt skalbar och kan klara av ett stor antal användare.\\
Ytterligare funktioner till själva appen tillkommer. Exempel på detta är:
\begin{itemize}
\item[•] Snygg layout för applikationen
\item[•] Vi funderar på att automatiskt kunna fylla på databasen med händelser från t.ex. Wikipedia genom att parsea olika samlingssidor.
\item[•] Uppkoppling till sociala medier
\item[•] Ranking
\item[•] Statistik
\item[•] Lista med vänner\\

Som tidigare nämnt kommer det även skapas:\\

\item[•] Klient
\item[•] Server
\end{itemize}

\section{Tekniska utmaningar}
\begin{enumerate}
\item Hur vi ska samordna mellan användare, vänner etc
\item Rating, matchmaking
\item Balansering av svårighetsgrad
\item Lära datorn att göra frågor själv (automatisk generering)
\item Att hantera feedback från Alpha/Beta-test
\item Att det faktiskt ska bli en \textbf{färdig produkt}

\subsection{Vad kan vi:}
\item[•] Använda oss av kunskaper från MDI, IOOPM, OSM \& PKD

\subsection{Vi vi kommer behöva lära oss:}
\item[•] Programmera för mobila enheter
\item[•] Att faktiskt ta fram en färdig produkt
\item[•] Programmera interface på ett bra sätt
\end{enumerate}

\section{Prioritetslista}
\begin{enumerate}
\item Server med grundläggande spelmekanik
\item Androidapplikation
\item[•] Snygg layout
\item[•] Fungera och snygg
\item Automatisk frågehämtare
\item Utveckla \textbf{servern} med: 
\item[•] Bättre matchmaking, Rankingsystem 
\item[•] Maskininlärning av svårighetsgrad på frågor

\end{enumerate}

\section{Balansering}

\begin{enumerate}
\item Tidsbegränsningen för spelaren
\item Extrapoäng
\item Svårighetsgrad
\item längd på omgång kontra "fortsättning" på omgång
\item Visa årtal eller inte.
\item[•] Alt 1: visa årtal vid förlorad omgång.
\item[•] Alt 2: Visa årtal efter varje rätt placerad händelse.
\item Räkna poäng under match:
\item[•] Total poäng under en omgång
\item[•] Räkna vinster under en omgång


\end{enumerate}

\end{document}
