\documentclass[11pt,a4paper]{article}
\usepackage[utf8]{inputenc}
\usepackage{amsmath}
\usepackage{amsfonts}
\usepackage{amssymb}
\author{FipparN}




\begin{document}

\section{Kontroll}
Om man ska prata om kontroll i applikationen så känner jag att kontrollen var min. Eftersom att appen inte hade så många knappar och alternativ så kände jag att det var lätt att navigera sig. Positivt att det inte fanns så många alternativ, det bidrar till att användaren lätt förstår vad hen ska trycka på för att komma till sitt mål. Det är inte helt solklart vart man ska trycka för att komma till de olika funktionerna appen ger. Men tack vare de få alternativen går det ganska snabbt ändå att navigera sig rätt.
"Plus"-ikonen är ganska straight forward men den kan betyda många saker vilket är positit i den meningen att det blir en ganska "clean" design. Att ha många alternativ samlade i en knapp betyder dock att en viss förvirring kan uppstå. Speciellt när den knappen bara har tre alternativ som skulle kunna finnas som separata knappar eller klickbara ikoner. 

\section{Synlighet}
Boxarna för matcher, rundor och korten som finns i en runda borde "poppa" mer från bakgrunden alternativt vara mer specifika. Detta skulle man kunna uppnå med färger, pilar eller liknande. Då skulle användaren lättare kunna koppla vart hen skulle komma om en speciell trycktes.\\
Informationen som ges för varje runda, exempelvis: ställningen i en runda och vilken omgång det är borde "poppa" mer från designen. Just nu är det för mycket text för att det ska vara enkelt för användaren att lätt ta in den värdefulla informationen. Förslagsvis borde mängden text som visas för varje runda minskas ner och ersättas med mer summerad information som står ut från bakgrunden. Detta skulle man kunna uppnå med en annan färgsättning eller om man vill experimentera med fet text etc. 
För att lösa problemet med vems tur det är skulle man kunna ha olika färger för varje runda. Varje färg betyder en sak, exempelvis: Grön = din tur, gul = motståndarens tur och röd = avslutad match. Om färgsättning inte är något som utvecklarna vill ändra på i matchvyn skulle de möjligtvis kunna använda sig av andra metoder för att särskilja matchernas status (vems tur det är etc). Utvecklarna skulle kunna ha en litet tomt utrymme mellan sektionerna "Din tur" och "Motståndarens tur" samt att se till att avslutade matcher mattades ur för att visa spelarna att dessa inte är aktiva.   
 

\section{Stil}
Stilen för applikationen var ren i den meningen att det inte var allt för mycket information på de olika menyerna. Färgsättningen i designen var mycket väl vald då det skapade bra kontraster för användaren vilket är otroligt viktigt. Flödet i applikationen var också väl designat då appens animationer var mjuka och lätta för öget vilket inte gör det jobbigt för spelaren att använda sig av applikationen.

\section{Om designen var konsekvent}
Applikationens design var konsekvent med formgivning och färgsättning av knappar och symboler. Dock finns det möjlighet att göra designen mer konsekvent. Exempel på detta var när man sökte efter spelare. Sökresultaten visades då som ett rutnät vilket inte riktigt följde den övriga strukturen för applikationens vyer. Istället för rutnät skulle sökresultatet visa varje spelare i en separat rad. Om denna design skulle väljas skulle utvecklarna även kunna få in ett antal knappar för att lägga till spelaren som vän, utmana spelare etc. Detta skulle bidra till att användaren slipper leta efter dessa funktioner.


\end{document}