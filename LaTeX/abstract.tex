\documentclass[10pt,a4paper]{article}
\usepackage[utf8]{inputenc}
\usepackage{amsmath}
\usepackage{amsfonts}
\usepackage{amssymb}
\title{Chronos}
\author{Alfred Yrelin, Josef Svensson, Philip Åkerfeldt}
\begin{document}
\maketitle
\begin{abstract}
Chronos är ett mobilspel utvecklat till Android som går ut på att rangordna historiska händelser bättre än motståndaren. Du kan spela mot antingen din kompis, eller mot någon okänd.
Varje spelare får en bunt händelser och den som kan relatera dessa till varandra och sätta dem i rätt ordning vinner. 

Chronos är inte bara ett roligt spel. Bakom användarens gränssnitt finns det ett intelligent system som anpassar spelet åt spelaren, för en maximalt rolig spelupplevelse. Spelet analyserar spelarens kunskapsnivå och anpassar händelserna i spelet för att matcha nivån. Serverapplikationen har också ett smart system för att automatiskt uppdatera spelet med nya frågor så att den erfarne spelaren aldrig får tråkigt.

Serverdelen är skriven i GO och använder Googles App Engine som plattform till stor del på grund av skalbarheten. Det intelligenta systemet balanserar frågorna och spelarna enligt Elo, ett system som används i många populära multiplayerspel.
\end{abstract}

\begin{abstract}
Chronos is a mobile game developed to the Android platform. The goal of the game is to arrange historical events in the correct order and to do it better than your opponent. You are able to play against either your friend or someone unknown. Each player is served a bunch of events and the one able to arrange them in most correct order is the winner.

Chronos is not just not a fun game. Behind the users interface there exists a intelligent system which adapts the game depending on the player, to get the most fun out of the game. The game analyses the players knowledge and adapts the questions against that. The server also has a system that automatically updates the database with new events so that the experienced player never gets bored. 

The server is written in the language GO and it uses Google App Engine as platform mostly due to its scalability. The intelligent system adapts the questions with Elo, which is a system used in many popular multiplayer games.
\end{abstract}
\end{document}