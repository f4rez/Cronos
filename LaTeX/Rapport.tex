\documentclass[12pt,a4paper]{article}
\usepackage[utf8]{inputenc}
\usepackage{amsmath}
\usepackage{amsfonts}
\usepackage{amssymb}
\author{Alfred Yrelin\\Josef Svensson\\Philip Åkerfeldt}
\title{Är du en historiemästare?}
\begin{document}
\maketitle
\begin{abstract}

\end{abstract}
\pagebreak
\section{Förutsättningar}
\subsection{Bakgrund}
Applikationsutveckling för mobila enheter är ett spännande område. För utveckling av själva applikationen krävs kunskaper om Java, XML och gränssnittsdesign, men det är inte allt som krävs i ett mer avancerat projekt. Många applikationer interagerar på något vis med internet eller med andra klienter. När sådan funktionalitet önskas så behövs det också kunskap om nätverkshantering och serverutveckling. I det här projektet sker också viss automatisk hämtning av information varvid kunskap om utveckling av ett smart system som filtrerar data behövs.
\subsection{Syfte}
Det är projektets syfte är att utveckla och distribuera en spelapplikation för mobila enheter. I det här projektet ska en applikation utvecklas som ger flera användare möjlighet att interagera med varandra.

Spelet som ska utvecklas går ut på att två användare får tävla om vem som har bäst koll på när olika historiska händelser ägde rum. Varje spelare får ett antal händelser som denne sedan ska sortera i kronologisk ordning. Spelaren som sorterar flest händelser rätt vinner. Förutom rent nöje så ger spelet också användaren möjlighet att få bättre koll på historien, applikationen kan alltså också användas i utbildningssammanhang.
\subsection{Mål}
Det finns tre stora delar i det här projektet som kan ses som projektets mål:
\begin{itemize}
\item En mobilapplikation
\item En server
\item Automatisk informationshämtning
\end{itemize}
\subsection{Motivation}
\subsection{Lägesbeskrivning}
\subsection{Frågeställningar}
\section{Genomförande}
\subsection{Metoder}
\subsection{Systemstruktur}
\subsection{Avgränsningar}
\subsection{Krav}
\subsection{Utvärdering}
\appendix
\section{Tidsplan}
\section{Kursmål}

\end{document}