\documentclass[12pt,a4paper]{article}
\usepackage[utf8]{inputenc}
\usepackage{amsmath}
\usepackage{amsfonts}
\usepackage{amssymb}
\author{Alfred Yrelin\\Josef Svensson\\Philip Åkerfeldt}
\title{Är du en historiemästare?}
\begin{document}
\maketitle
\begin{abstract}

\end{abstract}
\pagebreak
\section{Förutsättningar}
\subsection{Bakgrund}
Applikationsutveckling för mobila enheter är ett spännande område. För utveckling av själva applikationen krävs kunskaper om Java, XML och gränssnittsdesign, men det är inte allt som krävs i ett mer avancerat projekt. Många applikationer interagerar på något vis med internet eller med andra klienter. När sådan funktionalitet önskas så behövs det också kunskap om nätverkshantering och serverutveckling. I det här projektet sker också viss automatisk hämtning av information varvid kunskap om utveckling av ett smart system som filtrerar data behövs.
\subsection{Syfte}
Det är projektets syfte är att utveckla och distribuera en spelapplikation för mobila enheter. I det här projektet ska en applikation utvecklas som ger flera användare möjlighet att interagera med varandra.

Spelet som ska utvecklas går ut på att två användare får tävla om vem som har bäst koll på när olika historiska händelser ägde rum. Varje spelare får ett antal händelser som denne sedan ska sortera i kronologisk ordning. Spelaren som sorterar flest händelser rätt vinner. Förutom rent nöje så ger spelet också användaren möjlighet att få bättre koll på historien, applikationen kan alltså också användas i utbildningssammanhang.
\subsection{Mål}
Det finns tre stora delar i det här projektet som kan ses som projektets mål:
\begin{itemize}
\item En mobilapplikation
\item En server
\item Automatisk informationshämtning
\end{itemize}

\subsubsection{Mobilapplikation}
En mobilapplikation till först och främst Android ska utvecklas. Om det finns tid kommer också en version för ios att utvecklas. Mobilapplikationen är det enda som användaren i slutändan kommer se, därför är gränssnittet extra viktigt i den här delen. 

Applikationen till Android skrivs i Java och gränssnittet skrivs med hjälp av xml-markup.
\subsubsection{Server}
För att flera spelare ska kunna interagera med varandra så behövs det en server. En server gör också att frågorna som finns i spelet kommer kunna uppdateras kontinuerligt.

Servermjukvaran skrivs i programspråket Go och körs på Googles molntjänst App engine.

\subsubsection{Automatisk informationshämtning}
För att göra spelet så intressant som möjligt behövs det många frågor. Ett sätt att skapa många frågor är att utveckla mjukvara som kan hämta information och sätta ihop frågor automatiskt. 

Målet med den automatiska informationshämtningen är ett delsystem som läser in delar av Wikipedia och kopplar samman årtal och händelser automatiskt. Det här systemet behöver också ett administrativt gränssnitt så att en administratör på ett snabbt vis kan korrekturläsa frågorna innan de används.
\subsection{Motivation}
Det finns många mobilspel som handlar om att två personer möts och mäter sina färdigheter inom ett visst område. Exempel på applikationer är Quizkampen, Wordfeud och Wordbase. Det är tydligt att spel där två personer kan mötas för att vinna över den andre, genom att svara på frågor eller lägga ord, är populära. Det spel som den här rapporten handlar om kombinerar koncepten från tidigare nämnda spel, med att både svara på frågor och lägga in dem på rätt plats. 

En liknande version av det här spelet har också visat sig vara populär i sällskapsspelsform (\textit{När-då-då?}\footnote{http://www.spelexperten.com/sallskapsspel/familjespel/nar-da-da.html}, därför är hoppet högt om att också den här applikationen kan bli populär. Just nu finns det inte heller någon variant av det här spelkonceptet till Android, vilket är en lucka som kan täckas av det här projektet.
\subsection{Lägesbeskrivning}
Det finns ett spel för ios, \textit{Historiekampen}\footnote{https://itunes.apple.com/se/app/historiekampen/id840741173?mt=8} som också handlar om att lägga in händelser på en tidslinje. Genom en ytlig granskning av deras gränssnitt har det konstaterats att det går att göra bättre. Utan större kännedom om detaljerna i spelmekaniken så är också förhoppningen att den här applikationen ska bli roligare att spela. En nackdel som den befintliga ios-versionen har är att användaren endast ser årtal på sin tidslinje, och inte vilka händelser som ligger där.
\subsection{Frågeställningar}
\subsubsection{Tekniska problem}
Det finns vissa delar i det här systemet som blir mer avancerade att lösa än andra delar. Här listas några av dessa.
\begin{itemize}
\item Eftersom applikationen ska hämta information att göra frågor av automatiskt så måste ett sådant system konstrueras. En svårighet i detta är att se till så att systemet blir tillräckligt smart för att avgöra vad som kan användas som frågor och inte.
\item Det behövs en ganska stor databas för att hantera både användare och frågor. Det här är någonting som projektmedlemmarna inte har arbetat med i så stor skala förut.
\item Det krävs nätverkskommunikation med säkerhet och användaridentifiering. App Engine har användbara funktioner för delar av detta.
\item Spelet ska anpassa svårighetsgraden automatiskt så någon form av maskininlärning behövs. Tanken är att frågorna ska ha en svårighetsgrad som justeras automatiskt beroende på hur ofta de läggs rätt. Här måste det också tas hänsyn till hur resten av brädet ser ut vid tillfället.
\item Ett system för att rangordna spelare så att en topplista kan visas behövs.
\end{itemize}
\section{Genomförande}
\subsection{Metoder}
\begin{itemize}
\item Scrum: Projektet ska följa en agil struktur för programutveckling. Till hjälp används Trello\footnote{http://www.trello.com} för att skapa uppgifter enligt Scrums struktur. Kontinuerliga gruppmöten hålls också där lägesuppdateringar görs.
\item Android Studio används för androidutveckling och simulering.
\item Google App Engine används lokalt för simulering av serverapplikationen. I skarpt läge kommer också App Engine att användas, men då i Googles moln.
\item Wikipedia används som informationskälla för frågorna som spelet presenterar för användarna.
\item Git används för att synkronisera projektet. Ett privat repo på Bitbucket\footnote{http://www.bitbucket.org} används till detta.
\item Google Hangouts används för kommunikation när gruppmedlemmarna är på olika fysiska platser.
\end{itemize}
\subsection{Systemstruktur}
\begin{itemize}
\item Databas, i App Engine med biblioteket \textit{datastore}.
\item Server, körs i App Engine och skriven i Go.
\item Klient, Androidapplikation
\item Crawler, informationssökare, en del av servern.
\end{itemize}
\subsection{Avgränsningar}
En applikation för ios prioriteras inte i första hand. Skulle det bli tid över och resten av systemet anses tillräckligt klart så kommer detta att utvecklas, men annars inte. Någon applikation för Windows finns inte heller med i planeringen.
\subsection{Krav}
Systemet ska klara många spelare samtidigt, dock bara två per spel. Eftersom App Engine används för servern så kommer det inte bli något problem, det är en tjänst som automatiskt skalar beroende på belastning.

Informationssökaren ska tillsammans med den administrativa kontrollen ge frågor som håller hög kvalitet. Sökaren ska också vara så pass bra att den administrativa kontrollen går fort och att så lite som möjligt behöver korrigeras där.
\subsection{Utvärdering}
Planen är att en fungerande beta eller alfaversion av systemet ska färdigställas ett tag innan projektets slut. I den här versionen ska användare ha möjlighet att ge feedback och att testa applikationen. Så många fel som möjligt ska sedan justeras innan den skarpa versionen släpps.

Gränssnittet och användandet av applikationen kommer också att utvärderas med hjälp av metoder från Människa-data-interaktionsområdet.
\appendix
\section{Tidsplan}
\section{Kursmål}

\end{document}