\documentclass[a4paper,11pt]{article}
\usepackage{fullpage}

\usepackage[utf8]{inputenc}
\usepackage[british]{babel}

\usepackage{amsmath}
\usepackage{amssymb}
\usepackage{amsthm} \newtheorem{theorem}{Theorem}
\usepackage{color}
\usepackage{float}
\usepackage{listings}

\usepackage{multirow}
\usepackage{tikz} \usetikzlibrary{trees}
\usepackage{hyperref}  % should always be the last package


% useful (wrappers for) math symbols:
\newcommand{\Cardinality}[1]{\left\lvert#1\right\rvert}
%\newcommand{\Cardinality}[1]{\##1}
\newcommand{\Ceiling}[1]{\left\lceil#1\right\rceil}
\newcommand{\Floor}[1]{\left\lfloor#1\right\rfloor}
\newcommand{\Iff}{\Leftrightarrow}
\newcommand{\Implies}{\Rightarrow}
\newcommand{\Intersect}{\cap}
\newcommand{\Sequence}[1]{\left[#1\right]}
\newcommand{\Set}[1]{\left\{#1\right\}}
\newcommand{\SetComp}[2]{\Set{#1\SuchThat#2}}
\newcommand{\SuchThat}{\mid}
\newcommand{\Tuple}[1]{\langle#1\rangle}
\newcommand{\Union}{\cup}
\newcommand{\Returns}{\longrightarrow}


\title{\textbf{Självständigt arbete \\
    Uppsala University -- Spring 2015 \\
    To-Do list for project \\
    by Zinister Interactive}}              

\author{Alfred \and Josef \and Philip}

%\date{Month Day, Year}
\date{\today}

\begin{document}

\maketitle

\pagebreak

\section{För alla att fixa}
\begin{enumerate}
\item Ladda ner GO
\item Ladda App engine
\item Ladda ner Python
\item Android Studio - För appen
\item TexMaker - För rapport
\end{enumerate}

\section{Planera}
\begin{enumerate}
\item Kommer senare
\end{enumerate}

\section{Definiera för projektet}
\begin{enumerate}
\item Kommer senare
\end{enumerate}

\section{Regler för gruppen}
\begin{enumerate}
\item Pusha inget om det inte fungerar
\item Vid problem angående gruppen - Diskutera problemen strukturerat
\item Man måste komma på bokade möten. Anledningar att missa möten kan vara:
\item[•] Sjukdom
\item[•] Andra möten som ej kan bokas om
\item[•] Krig
\item[•] Överenskommelse genom gruppen
\item[•] Communication is key.
\item Jobba enligt mallen för Agile Development

\end{enumerate}

\section{Tekniska utmaningar}
\begin{enumerate}
\item Hur vi ska samordna mellan användare, vänner etc
\item Rating, matchmaking
\item Balansering av svårighetsgrad
\item Lära datorn att göra frågor själv (automatisk generering)
\item Att hantera feedback från Alpha/Beta-test
\item Att det faktiskt ska bli en \textbf{färdig produkt}

\subsection{Vad kan vi:}
\item[•] Använda oss av kunskaper från MDI, IOOPM, OSM \& PKD

\subsection{Vi vi kommer behöva lära oss:}
\item[•] Programmera för mobila enheter
\item[•] Att faktiskt ta fram en färdig produkt
\item[•] Programmera interface på ett bra sätt
\item[•] 

\end{enumerate}

\section{Prioritetslista}
\begin{enumerate}
\item Server med grundläggande spelmekanik
\item Androidapplikation
\item[•] Snygg layout
\item[•] Fungera och snygg
\item Automatisk frågehämtare
\item Utveckla \textbf{servern} med: 
\item[•] Bättre matchmaking, Rankingsystem 
\item[•] Maskininlärning av svårighetsgrad på frågor

\end{enumerate}

\section{Spelregler}
\begin{enumerate}
\item Man spelar en mot en eller en mot comp
\item Varje match har ett game-mode
\item[•] Först till en viss poäng
\item[•] Vinna flest rundor under en match
\item[•] Infinity mode
\item[•] Rapid mode
\item Båda personerna får samma historiska händelse som referens (start) vid början på en omgång
\item Varje person får minst 5 händelser (samma händelser som motståndaren) som den ska placera ut på tidslinjen
\item[•] Om man placerar första händelsen fel så får man inte placera ut dom andra händelserna och den rundan är därmed över för spelaren
\item[•] Om användaren placerar ut händelsen på rätt plats kommer den få fortsätta att placera ut nästa händelse och så vidare.
\item Om antal frågor är exempelvis 5 och användaren får 5/5 rätt så kommer den få 5p + X st extrapoäng eller liknande.
\item   

\end{enumerate}

\section{Balansering}

\begin{enumerate}
\item Tidsbegränsningen för spelaren
\item Extrapoäng
\item Svårighetsgrad
\item längd på omgång kontra "fortsättning" på omgång
\item Visa årtal eller inte.
\item[•] Alt 1: visa årtal vid förlorad omgång.
\item[•] Alt 2: Visa årtal efter varje rätt placerad händelse.
\item Räkna poäng under match:
\item[•] Total poäng under en omgång
\item[•] Räkna vinster under en omgång
\end{enumerate}


\section{Att fixa lite grann till inlämning 2}

\begin{enumerate}
\item Teknisk 
\item Genomförande (under metoder)
\item ABSTRAKT svepande formuleringar skall undvikas ("sedan tidernas begynnelse")
\item METOD reproducerbarhet är fokus
\item AVGRÄNSNINGAR finns det flera avgränsningar? Hur många frågor ska vi hantera? etc
\item KRAV hantera många spelare, är det ens ett krav? Frågor ska ha hg kvalitet, vad betyder det? Utveckla denna kategori såklart.
\item UTVÄRDERING mot kraven och mot målen.
\item Referenser

\item PRIO: Bakgrund och relaterat arbete
\item[•] Brett perspektiv
\item[•] Spel och crawlers. Pederssens rapport
\item Börja skriv på systemstrukturen



\end{enumerate}

\end{document}
